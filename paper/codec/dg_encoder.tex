%!TEX root = ../paper_ecg_cs_codec.tex
\begin{tikzpicture}[
scale=0.6, every node/.style={transform shape},
node distance=10mm and 10mm,
% basic style
cell/.style = {
    align=center,
    minimum width=2cm,
    minimum height=1cm
},
% normal blocks
block/.style= {cell, draw, shape=rectangle},
]
% Write your diagram code here
\node[cell](src){Digital\\Samples};
\node[block, right=of src](win){Windowing};
\node[block, right=of win](cs){Compressive\\Sensing};
\node[block, right=of cs](fl){Flattening};
\node[block, below=of win](q){Adaptive\\Quantization};
\node[block, right=of q](cl){Adaptive\\Clipping};
\node[block, right=of cl](ec){Entropy\\Coding};
\node[cell, right=of ec](cb){Compressed\\Bits};

% paths
\path[draw,->]
(src) edge (win)
(win) edge (cs)
(cs) edge (fl)
(q) edge (cl)
(cl) edge (ec)
(ec) edge (cb);

\draw[->] (fl.south) -- +(0, -3mm) -| (q.north);
\end{tikzpicture}

